%% skmath improved math commands
%%
%% Copyright (C) 2012-2013 by Simon Sigurdhsson <sigurdhsson@gmail.com>
%% 
%% This work may be distributed and/or modified under the
%% conditions of the LaTeX Project Public License, either version 1.3
%% of this license or (at your option) any later version.
%% The latest version of this license is in
%%   http://www.latex-project.org/lppl.txt
%% and version 1.3 or later is part of all distributions of LaTeX
%% version 2005/12/01 or later.
%% 
%% This work has the LPPL maintenance status `maintained'.
%% 
%% The Current Maintainer of this work is Simon Sigurdhsson.
%% 
%% This work consists of the file skmath.tex
%% and the derived file skmath.sty.
\PassOptionsToPackage{intlimits}{amsmath}
\documentclass[commonsets,load]{skdoc}
\usepackage[intlimits]{amsmath}
\usepackage{csquotes}
\ProvideDocumentCommand\d{m}{#1} 
\ProvideDocumentCommand\abs{m}{#1} 
\ProvideDocumentCommand\norm{m}{#1} 
\ProvideDocumentCommand\abs{m}{#1} 
\ProvideDocumentCommand\P{m}{#1}
\ProvideDocumentCommand\given{}{} 
\ProvideDocumentCommand\E{m}{#1} 
\ProvideDocumentCommand\var{m}{#1} 
\ProvideDocumentCommand\cov{mm}{#1#2} 
\ProvideDocumentCommand\N{}{} 
\ProvideDocumentCommand\Z{}{} 
\ProvideDocumentCommand\Q{}{} 
\ProvideDocumentCommand\R{}{} 
\ProvideDocumentCommand\C{}{} 
\ProvideDocumentCommand\sfrac{mm}{#1#2} 

% Declare the target files
\SelfPreambleTo{\mypreamble}
\DeclareFile[key=package,preamble=\mypreamble]{skmath.sty}

% This is where the documentation begins
\begin{document}
  % Change & version info
  \version{0.1h}
  \changes{0.1}{Initial version}
  \changes{0.1c}{Moved package from \pkg{docstrip} to \pkg{skdoc}}
  \changes{0.1d}{Fixed fatal documentation and package errors}
  \changes{0.1e}{Added statistics commands}
  \changes{0.1g}{Documentation fixes}

  % Metadata
  \package[ctan=skbundle,vcs=https://github.com/urdh/skmath]{skmath}
  \author{Simon Sigurdhsson}
  \email{sigurdhsson@gmail.com}

  % First page
  \maketitle
  \begin{abstract}
    The \thepackage\ package provides improved and new math commands
    for superior typesetting with less effort.
  \end{abstract}

  \section{Introduction}
  This package intends to provide helpful (re-)definitions of commands
  related to typesetting mathematics, and specifically typesetting
  them in a more intuitive, less verbose and more beautiful way.
  It was originally not intended for use by the public, and as such
  there may be incompatibilities with other packages of which I am
  not aware, but I figured it could be useful to other people as well.

  \section{Usage}
  \subsection{Options}
  As of version \theversion, there is only one option: \opt{commonsets}.
  By default, it is disabled but if the option is given the package will
  define \cs{N}, \cs{Z}, \cs{Q}, \cs{R} and \cs{C} as blackboard 
  variants of the respective letters, to represent the common sets
  of numbers.

  \subsection{New commands}
  The package defines a number of new commands that aid in typesetting
  certain mathematical formulae.
 
  \DescribeMacro\N 
  \DescribeMacro\Z 
  \DescribeMacro\Q 
  \DescribeMacro\R 
  \DescribeMacro\C
  These commands are only available if the \opt{commonsets}
  option is given. They typeset the set of natural, integer, rational,
  real and complex numbers respectively.
  \begin{example}
    \begin{equation*}
      \N, \Z, \Q, \R, \C.
    \end{equation*}
  \end{example}
 
  \DescribeMacro\norm{<expression>}
  \DescribeMacro\abs{<expression>}
  The commands \Macro\norm and \Macro\abs, quite expectedly, typeset
  the norm ans absolute value of an expression, respectively. They
  have one mandatory argument (the expression), and different norms
  can be achieved by appending a subscript after the argument of 
  \Macro\norm.
  \begin{example}
    \begin{equation*}
      \norm{\vec{x}}_p = \left(\sum_{i=1}^n \abs{x_i}^p\right)^{\sfrac{1}{p}}
    \end{equation*}
  \end{example}
 
  \DescribeMacro\d{<variable>}
  There is also a command \Macro\d, with one mandatory argument, that 
  typesets the differential part of an integral.
  \begin{example}
    \begin{equation*}
      \int_{\R}\! \frac{\sin{x}}{x} \d{x}
    \end{equation*}
  \end{example}

  \DescribeMacro\E{<expression>}
  The command \Macro\E typesets the expectation of a random variable.
  \begin{example}
    \begin{equation*}
      \E{\hat{\mu}} = \mu
    \end{equation*}
  \end{example}

  \DescribeMacro\P{<expression>\AlsoMacro\given <expression>}
  The \Macro\P command typesets a probability. The \Macro\given command 
  can be used to typeset conditional probabilities, within \Macro\P.
  \begin{example}
    \begin{equation*}
      \P{A\given B} = \frac{\P{B\given A}\P{A}}{\P{B}}
    \end{equation*}
  \end{example} 

  \DescribeMacro\var{<expression>}
  \DescribeMacro\cov{<expression>}{<expression>}
  The commands \Macro\var and \Macro\cov typeset the variance and
  covariance of an expression.
  \begin{example}
    \begin{gather*}
      \var{X} = \E{(X-\mu)^2}\\
      \cov{X}{Y} = \E{XY}-\E{X}\E{Y}
    \end{gather*}
  \end{example}
 
  \subsection{Improved commands}
  In addition to adding new commands, this package also redefines
  already existing commands in a mostly backwards-compatible way
  to improve their usefulness.
 
  \DescribeMacro\sin[<power>]{<expression>} 
  \DescribeMacro\arcsin{<expression>} 
  \DescribeMacro\cos[<power>]{<expression>} 
  \DescribeMacro\arccos{<expression>} 
  \DescribeMacro\tan[<power>]{<expression>} 
  \DescribeMacro\arctan{<expression>} 
  \DescribeMacro\cot[<power>]{<expression>} 
  The trigonometric functions have been redefined
  to typeset more easily. They typeset \meta{expression} as an
  argument of the expression, and (if applicable) \meta{power} as
  a superscript between the function and its argument,
  \emph{e.g.} \(\sin[2]{\phi}\).
 
  \DescribeMacro\ln{<expression>}
  The natural logarithm macro \Macro\ln has also been redefined to 
  require an argument which is typeset as the argument of the logarithm.
  \DescribeMacro\log[<base>]{<expression>}
  The related macro \Macro\log is redefined in a similar way, but also 
  accepts an optional argument denoting the base of the logarithm:
  \(\log[2]{x}\).
 
  \DescribeMacro\exp{<expression>}
  The exponential, \Macro\exp, is redefined to typeset its argument as a
  superscript of \(e\) in some display styles, and as an argument of
  \(\mathrm{exp}\) otherwise:
  \begin{equation*}
    \exp{\sqrt{2}\exp{x}}
  \end{equation*}

  \subsection{Stylistic changes}
  Some commands have been redefined in a completely backwards-compatible
  way to improve the end result of their typesetting.
 
  \DescribeMacro\frac{<numerator>}{<denominator>}
  The \Macro\frac command has been changed to improve typesetting,
  allowing displaystyle math in some settings.
 
  \DescribeMacro\bar{<expression>}
  \DescribeMacro\vec{<expression>}
  The \Macro\bar command has been changed to cover the entire 
  \meta{expression} (\emph{i.e.} \(\bar{uv}\)), and \Macro\vec has
  been changed to match the \cs{vectorsym} command provided by 
  \pkg{isomath}.

  \section{Known issues}
  A list of current issues is available in the Github repository of this
  package\footnote{\url{https://github.com/urdh/skmath/issues}}, but as
  of the release of \theversion, there is only one known issue:
  \begin{description}
      \item[\#4]  When using both \pkg{fontspec} and \thepackage, sometimes
                  \LaTeX\ bails out saying that \enquote{\cs{bar} is already
                  defined}. This is probably due to fonts defining their own
                  \cs{bar}, and will happen with other font packages as well,
                  but I haven't figured out a suitable solution yet.
  \end{description}
  If you discover any bugs in this package, please report them to the issue
  tracker in the \thepackage\ Github repository.

  \section{Implementation}
  The package implementation is very simple. First, we do the standard
  \LaTeXe\ preamble thing, then we require some dependencies.
  \changes{0.1b}{Load \textsf{amsmath} with \texttt{intlimits} option}
\begin{MacroCode}{package}
\NeedsTeXFormat{LaTeX2e}[1999/12/01]
\ProvidesPackage{skmath}%
    [2013/03/24 v0.1h skmath improved math commands]
\RequirePackage{xparse}
\PassOptionsToPackage{intlimits}{amsmath}
\RequirePackage{kvoptions,amssymb,mathtools,xfrac,isomath}
\end{MacroCode}

  We begin by declaring an option.
\begin{MacroCode}{package}
\SetupKeyvalOptions{family=skmath,prefix=skmath@}
\DeclareBoolOption[false]{commonsets}
\ProcessKeyvalOptions*
\end{MacroCode}

  We optionally provide commands to typeset common sets.
\begin{MacroCode}{package}
\ifskmath@commonsets
\end{MacroCode}
  \begin{macro}{\N}
  \changes{0.1b}{Moved to \textsf{xparse} command definition}
\begin{MacroCode}{package}
  \NewDocumentCommand\N{}{\ensuremath{\mathbb{N}}}
\end{MacroCode}
  \end{macro}
  \begin{macro}{\Z}
  \changes{0.1b}{Moved to \textsf{xparse} command definition}
\begin{MacroCode}{package}
  \NewDocumentCommand\Z{}{\ensuremath{\mathbb{Z}}}
\end{MacroCode}
  \end{macro}
  \begin{macro}{\Q}
  \changes{0.1b}{Moved to \textsf{xparse} command definition}
\begin{MacroCode}{package}
  \NewDocumentCommand\Q{}{\ensuremath{\mathbb{Q}}}
\end{MacroCode}
  \end{macro}
  \begin{macro}{\R}
  \changes{0.1b}{Moved to \textsf{xparse} command definition}
\begin{MacroCode}{package}
  \NewDocumentCommand\R{}{\ensuremath{\mathbb{R}}}
\end{MacroCode}
  \end{macro}
  \begin{macro}{\C}
  \changes{0.1b}{Moved to \textsf{xparse} command definition}
\begin{MacroCode}{package}
  \NewDocumentCommand\C{}{\ensuremath{\mathbb{C}}}
\end{MacroCode}
  \end{macro}
\begin{MacroCode}{package}
\fi
\end{MacroCode}
 
  This is followed by commands to typeset the norm and absolute value.
  \begin{macro}{\abs}
\begin{MacroCode}{package}
\DeclarePairedDelimiter\abs{\lvert}{\rvert}
\end{MacroCode}
  \end{macro}
  \begin{macro}{\norm}
\begin{MacroCode}{package}
\DeclarePairedDelimiter\norm{\lVert}{\rVert}
\end{MacroCode}
  \end{macro}

  Next come the statistical commands.
  \begin{macro}{\E}
  \changes{0.1e}{Added \cs{E} command}
  \changes{0.1f}{Fixed \enquote{Command \cs{E} already defined!} error}
  Here, we define \cs{E} after the preamble since it may break otherwise.
\begin{MacroCode}{package}
\AtBeginDocument{
  \DeclareDocumentCommand\E{m}{%
    \ensuremath{\mathop{\mathrm{E}}\left[#1\right]}%
  }
}
\end{MacroCode}
  \end{macro}
  The \Macro\P command saves any old \Macro\given command, replacing
  it locally with the new \Macro\given command provided by the package.
  \begin{macro}{\P}
  \changes{0.1e}{Added \cs{P} command}
\begin{MacroCode}{package}
\DeclareDocumentCommand\P{m}{%
  \ensuremath{\mathop{\mathrm{P}}%
    \left(%
    \let\skmath@given\given%
\end{MacroCode}
  \begin{macro}{\given}
  \changes{0.1e}{Added \cs{given} command}
\begin{MacroCode}{package}
    \DeclareDocumentCommand\given{}{\mid}%
\end{MacroCode}
  \end{macro}
\begin{MacroCode}{package}
    #1%
    \let\given\skmath@given%
    \right)%
  }%
}
\end{MacroCode}
  \end{macro}
  \begin{macro}{\var}
  \changes{0.1e}{Added \cs{var} command}
\begin{MacroCode}{package}
  \DeclareDocumentCommand\var{m}{%
    \ensuremath{\mathop{\mathrm{Var}}\left(#1\right)}%
  }
\end{MacroCode}
  \end{macro}
  \begin{macro}{\cov}
  \changes{0.1e}{Added \cs{cov} command}
\begin{MacroCode}{package}
  \DeclareDocumentCommand\cov{mm}{%
    \ensuremath{\mathop{\mathrm{Cov}}\left(#1,#2\right)}%
  }
\end{MacroCode}
  \end{macro}
  
  We replace all trigonometric functions and some other
  common functions with alternatives that take an argument
  (or optionally, several arguments).
\begin{MacroCode}{package}
\let\skmath@sin\sin
\let\skmath@cos\cos
\let\skmath@tan\tan
\let\skmath@cot\cot
\let\skmath@arcsin\arcsin
\let\skmath@arccos\arccos
\let\skmath@arccos\arctan
\let\skmath@ln\log
\let\skmath@log\log
\let\skmath@exp\exp
\end{MacroCode}
  \begin{macro}{\sin}
\begin{MacroCode}{package}
\RenewDocumentCommand\sin{om}{%
  \IfNoValueTF{#1}
    {\ensuremath{\skmath@sin\left(#2\right)}}
    {\ensuremath{\skmath@sin^{#1}\left(#2\right)}}%
}
\end{MacroCode}
  \end{macro}
  \begin{macro}{\cos}
\begin{MacroCode}{package}
\RenewDocumentCommand\cos{om}{%
  \IfNoValueTF{#1}
    {\ensuremath{\skmath@cos\left(#2\right)}}
    {\ensuremath{\skmath@cos^{#1}\left(#2\right)}}%
}
\end{MacroCode}
  \end{macro}
  \begin{macro}{\tan}
\begin{MacroCode}{package}
\RenewDocumentCommand\tan{om}{%
  \IfNoValueTF{#1}
    {\ensuremath{\skmath@tan\left(#2\right)}}
    {\ensuremath{\skmath@tan^{#1}\left(#2\right)}}%
}
\end{MacroCode}
  \end{macro}
  \begin{macro}{\cot}
\begin{MacroCode}{package}
\RenewDocumentCommand\cot{om}{%
  \IfNoValueTF{#1}
    {\ensuremath{\skmath@cot\left(#2\right)}}
    {\ensuremath{\skmath@cot^{#1}\left(#2\right)}}%
}
\end{MacroCode}
  \end{macro}
  \begin{macro}{\arcsin}
\begin{MacroCode}{package}
\RenewDocumentCommand\arcsin{m}{%
  \ensuremath{\skmath@arcsin\left(#1\right)}%
}
\end{MacroCode}
  \end{macro}
  \begin{macro}{\arccos}
\begin{MacroCode}{package}
\RenewDocumentCommand\arccos{m}{%
  \ensuremath{\skmath@arccos\left(#1\right)}%
}
\end{MacroCode}
  \end{macro}
  \begin{macro}{\arctan}
\begin{MacroCode}{package}
\RenewDocumentCommand\arctan{m}{%
  \ensuremath{\skmath@arctan\left(#1\right)}%
}
\end{MacroCode}
  \end{macro}
  \begin{macro}{\ln}
\begin{MacroCode}{package}
\RenewDocumentCommand\ln{m}{%
  \ensuremath{\skmath@ln\left(#1\right)}%
}
\end{MacroCode}
  \end{macro}
  \begin{macro}{\log}
\begin{MacroCode}{package}
\RenewDocumentCommand\log{om}{%
  \IfNoValueTF{#1}
    {\ensuremath{\skmath@log\left(#2\right)}}
    {\ensuremath{\skmath@log_{#1}\left(#2\right)}}%
}
\end{MacroCode}
  \end{macro}
  \begin{macro}{\exp}
  \changes{0.1b}{Moved to \textsf{xparse} command definition}
\begin{MacroCode}{package}
\RenewDocumentCommand\exp{m}{\ensuremath{\mathchoice%
  {e^{#1}}%
  {\skmath@exp\left(#1\right)}%
  {\skmath@exp\left(#1\right)}%
  {\skmath@exp\left(#1\right)}%
}}
\end{MacroCode}
  \end{macro}

  The fraction command is modified to improve typesetting.
  \begin{macro}{\frac}
  \changes{0.1b}{Moved to \textsf{xparse} command definition}
\begin{MacroCode}{package}
\RenewDocumentCommand\frac{mm}{\genfrac{}{}{}{}%
             {\displaystyle #1}{\displaystyle #2}}
\end{MacroCode}
  \end{macro}
 
  The \cs{bar} command is also modified to impove typesetting.
  \begin{macro}{\bar}
  \changes{0.1b}{Added \cs{bar} replacement}
\begin{MacroCode}{package}
\RenewDocumentCommand\bar{m}{%
    \ensuremath{\mkern 1.5mu\overline{\mkern-1.5mu{#1}\mkern-1.5mu}\mkern 1.5mu}}
\end{MacroCode}
  \end{macro}

  We introduce a command to typeset the differential part
  of integrals, shamefully stolen from an answer on \TeX.SE.
  Definition is deferred until after all packages are loaded
  to avoid collisions with other \cs{d} commands.
\begin{MacroCode}{package}
\AtBeginDocument{%
\end{MacroCode}
  \begin{macro}{\d}
  \changes{0.1a}{Fixed obtuse errors}
  \changes{0.1b}{Moved to \textsf{xparse} command definition}
\begin{MacroCode}{package}
\DeclareDocumentCommand\d{m}{\ensuremath{\,\mathrm{d}#1%
                              \@ifnextchar\d{\!}{}}}
\end{MacroCode}
  \end{macro}
\begin{MacroCode}{package}
}
\end{MacroCode}
 
  Finally, we define a nicer way to denote vectors.
  \begin{macro}{\vec}
\begin{MacroCode}{package}
\let\vec\vectorsym
\end{MacroCode}
  \end{macro}
\begin{MacroCode}{package}
\endinput
\end{MacroCode}

    \PrintChanges
    \PrintIndex
\end{document}
